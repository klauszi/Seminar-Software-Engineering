\documentclass[runningheads]{llncs}

%---- Sonderzeichen-------%
\usepackage {ngerman}
%---- Codierung----%
\usepackage[latin1]{inputenc}	% for Unix and Windows
\usepackage[T1]{fontenc}
\usepackage{graphicx}
\usepackage{url}
\usepackage{llncsdoc}
%----- Mathematischer Zeichenvorrat---%
\usepackage{amsmath}
\usepackage{amssymb}
\usepackage{enumerate}
% fuer die aktuelle Zeit
\usepackage{scrtime}
\usepackage{listings}
\usepackage{subfigure}
\usepackage{hyperref}

\setcounter{tocdepth}{3}
\setcounter{secnumdepth}{3}



\begin{document}

\mainmatter
\title{Titel der Seminararbeit}
\titlerunning{Titel der Seminararbeit}
\author{Name des Autors}
\authorrunning{Titel des Seminars}
\institute{Betreuer: Betreuername}
\date{23.07.2007}
\maketitle

\begin{abstract} An dieser Stelle sollte sp�ter eine Kurzzusammenfassung stehen.
        \cite{Lan07}
\end{abstract}

\section{Kapitel}
\subsection{Unterkapitel}
% Normaler LNCS Zitierstil
%\bibliographystyle{splncs}
\bibliographystyle{itmalpha}
% TODO: �ndern der folgenden Zeile, damit die .bib-Datei gefunden wird
\bibliography{../Literatur/lit}

\end{document}

